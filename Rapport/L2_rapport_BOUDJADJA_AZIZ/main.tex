\documentclass[a4paper, 12pt, twoside]{article}
\usepackage[utf8]{inputenc}
\usepackage[T1]{fontenc}
\usepackage[francais]{babel}
\usepackage{lmodern}
\usepackage{ae,aecompl}
\usepackage[top=2.5cm, bottom=2.5cm, left=3cm, right=2.5cm, headheight=15pt]{geometry}
\usepackage{graphicx}
\usepackage{eso-pic}
\usepackage{array}
\usepackage{hyperref}
\input{pagedegarde}  % Fichier obligatoire pour la page de garde personnalisée

\title{Prédiction des prix des actions du S\&P 500 à l'aide de Machine Learning}
\entreprise{}  % Pas d'entreprise (projet académique)
\datedebut{10/2025}  % Ex. : 1 septembre 2025
\datefin{12/2025}      % Ex. : 20 décembre 2025
\membrea{BOUDJADJA Anis (45000199) -- \url{https://github.com/Anis017/SP500-Prediction-}}
\membreb{Aziz Amor     (44020039)}

% Commentez les lignes suivantes si seulement deux membres

\begin{document}

\pagedegarde

\section*{Remerciements}
Nous tenons à remercier notre encadrant pour ses conseils précieux, ainsi que les outils open-source qui ont facilité le développement de ce projet.

\newpage
\tableofcontents
\newpage

\section{Introduction}
Ce projet consiste en la réalisation d'une application web permettant de prédire les prix futurs des actions composant l'indice S\&P 500 à l'aide de techniques de Machine Learning, plus précisément un modèle de séries temporelles autorégressif (AutoReg).

L'objectif principal est de fournir aux utilisateurs un outil intuitif pour visualiser les données historiques et obtenir des prévisions à court et moyen terme (5 à 180 jours), tout en soulignant les limites inhérentes à la prédiction boursière.

Le projet s'inscrit dans le cadre d'un travail académique en informatique, en mettant l'accent sur l'utilisation de bibliothèques Python modernes pour le traitement de données financières.

\section{Environnement de travail}
Le développement a été réalisé en Python 3, avec un environnement virtuel pour gérer les dépendances. L'application est déployée localement via Streamlit, qui permet une interface web interactive sans serveur complexe.

Les outils principaux utilisés :
\begin{itemize}
\item Éditeur : VS Code ou PyCharm.
\item Gestion de versions : Git et GitHub (dépôt public : \url{https://github.com/Anis017/SP500-Prediction-})
\item Exécution : Streamlit pour l'interface utilisateur.
\end{itemize}

\section{Description du projet et objectifs}
\subsection{Problématique}
La prédiction des prix des actions est un défi majeur en raison de la nature volatile et influencée par de nombreux facteurs externes des marchés financiers. L'indice S\&P 500 regroupe 500 grandes entreprises américaines, et ses composants sont souvent analysés pour des décisions d'investissement.

\subsection{Objectifs}
\begin{itemize}
\item Récupérer des données historiques en temps réel via l'API Yahoo Finance.
\item Appliquer un modèle AutoReg pour prévoir les prix futurs.
\item Fournir une interface utilisateur conviviale avec graphiques interactifs et export CSV.
\item Limiter les prévisions à des horizons raisonnables (jusqu'à 180 jours).
\end{itemize}

\section{Bibliothèques, Outils et technologies}
Le projet repose sur les bibliothèques suivantes :
\begin{itemize}
\item \textbf{Streamlit} : Interface web interactive.
\item \textbf{yfinance} : Récupération des données boursières.
\item \textbf{statsmodels} : Implémentation du modèle AutoReg.
\item \textbf{Plotly} : Graphiques interactifs.
\item \textbf{Pandas} : Manipulation des données.
\end{itemize}

La structure du projet est la suivante :
\begin{verbatim}
SP500-PREDICTION-
├── assets/
│   └── data/
│       └── sp500_tickers.csv
├── streamlit_app/
│   ├── modules/
│   │   └── helper.py
│   ├── pages/
│   │   └── 01_��_StockPredictor.py
│   └── 00_ℹ️_Info.py
├── LICENSE
├── README.md
└── requirements.txt
\end{verbatim}

\section{Travail réalisé}
Les fonctionnalités prévues et réalisées :
\begin{itemize}
\item Sélection d'un ticker parmi les entreprises du S\&P 500 : réalisée.
\item Récupération des données historiques (2 ans) : réalisée.
\item Entraînement du modèle AutoReg et prévision (5 à 180 jours) : réalisée.
\item Affichage interactif avec Plotly (historique + prévision) : réalisée.
\item Téléchargement des prévisions en CSV : réalisée.
\item Interface multilingue/informations : réalisée via pages Streamlit.
\end{itemize}

Aucune fonctionnalité majeure n'a été abandonnée.

\textbf{Répartition du travail :}
\begin{itemize}
\item Amor : Développement de l'interface Streamlit et intégration Plotly.
\item Anis : Implémentation du modèle AutoReg, gestion des données yfinance et tests.
\end{itemize}

\section{Aide apportée par l'intelligence artificielle}
L'intelligence artificielle (notamment Grok de xAI et/ou ChatGPT) a été utilisée à plusieurs étapes :
\begin{itemize}
\item Génération et optimisation de code (ex. : traitement des données, débogage yfinance).
\item Aide conceptuelle pour le choix et le paramétrage du modèle AutoReg.
\item Rédaction et structuration de parties du rapport.
\item Suggestions pour l'amélioration de l'interface Streamlit.
\end{itemize}
Cela a accéléré le développement tout en approfondissant la compréhension des concepts.

\section{Difficultés rencontrées}
\begin{itemize}
\item Volatilité des marchés rendant les prévisions imprécises à long terme.
\item Choix optimal des lags dans le modèle AutoReg.
\item Gestion des données manquantes (jours fériés).
\item Performances variables selon la liquidité des tickers.
\end{itemize}

\section{Démonstration du projet}

Cette section présente les captures d'écran de l'application en action, montrant les différentes fonctionnalités et l'interface utilisateur.

\subsection{Page d'accueil}
\begin{center}
\includegraphics[width=0.8\textwidth]{télécharger (1).png}
\captionof{figure}{Page d'accueil de l'application avec les informations du projet}
\end{center}

\subsection{Sélection et visualisation des prédictions}
\begin{center}
\includegraphics[width=0.8\textwidth]{télécharger (2).png}
\captionof{figure}{Interface de sélection du ticker et graphique de prédiction}
\end{center}

\subsection{Export et résultats}
\begin{center}
\includegraphics[width=0.8\textwidth]{télécharger.png}
\captionof{figure}{Affichage des résultats et options d'export en CSV}
\end{center}

\section{Bilan}
\subsection{Conclusion}
Le projet a permis de développer une application fonctionnelle démontrant l'application du Machine Learning aux données financières. Tous les objectifs principaux ont été atteints.

\subsection{Perspectives}
\begin{itemize}
\item Intégrer des modèles plus avancés (LSTM, Prophet).
\item Ajouter des indicateurs techniques (RSI, MACD).
\item Déploiement en ligne.
\item Analyse de sentiment à partir de news.
\end{itemize}

\newpage
\section{Bibliographie}
\begin{thebibliography}{9}
\bibitem{yfinance} Yahoo Finance API documentation, \url{https://pypi.org/project/yfinance/}
\bibitem{statsmodels} Statsmodels documentation, AutoReg model, \url{https://www.statsmodels.org/stable/generated/statsmodels.tsa.ar_model.AutoReg.html}
\bibitem{streamlit} Streamlit documentation, \url{https://docs.streamlit.io/}
\end{thebibliography}

\newpage
\section{Webographie}
\begin{thebibliography}{9}
\bibitem{sp500} Liste des tickers S\&P 500, Wikipedia.
\bibitem{plotly} Plotly Python, \url{https://plotly.com/python/}
\end{thebibliography}

\newpage
\section{Annexes}
\appendix
\makeatletter
\def\@seccntformat#1{Annexe~\csname the#1\endcsname:\quad}
\makeatother

\section{Cahier des charges}
Description détaillée des fonctionnalités prévues (voir section ``Travail réalisé'').

\section{Exemple d'exécution du projet}
L'application est exécutée localement via la commande :
\begin{verbatim}
streamlit run streamlit_app/00_ℹ️_Info.py
\end{verbatim}

L'utilisateur peut alors :
\begin{enumerate}
\item Accéder à la page d'information.
\item Sélectionner un ticker S\&P 500 dans la liste déroulante.
\item Choisir l'horizon de prévision (5 à 180 jours).
\item Visualiser le graphique interactif montrant les données historiques et les prévisions.
\item Télécharger les résultats sous forme de fichier CSV.
\end{enumerate}

\section{Manuel utilisateur}
\begin{enumerate}
\item Cloner le dépôt GitHub.
\item Créer et activer un environnement virtuel : \verb|python -m venv .venv|
\item Installer les dépendances : \verb|pip install -r requirements.txt|
\item Se placer dans le dossier \verb|streamlit_app|
\item Lancer l'application : \verb|streamlit run 00_ℹ️_Info.py|
\item L'application s'ouvre dans le navigateur à l'adresse \url{http://localhost:8501}
\end{enumerate}

\end{document}